%----------------------------------------------------------------------------------------
%	PACKAGES AND OTHER DOCUMENT CONFIGURATIONS
%----------------------------------------------------------------------------------------
\nonstopmode
\documentclass[12pt,fleqn]{book} % Default font size and left-justified equations
\renewcommand*{\rmdefault}{Roboto}
%----------------------------------------------------------------------------------------
\input{structure} % Insert the commands.tex file which contains the majority of the structure behind the template
%\pagestyle{fancy}

\begin{document}	

%----------------------------------------------------------------------------------------
%	TITLE PAGE
%----------------------------------------------------------------------------------------
~\\
\begin{figure}

\end{figure}

\begingroup
\thispagestyle{empty}
\begin{tikzpicture}[remember picture,overlay]
  \coordinate [below=12cm] (midpoint) at (current page.north);
  \node at (current page.north west)
  {\begin{tikzpicture}[remember picture,overlay]
      %\node[anchor=north west,inner sep=0pt] at (0,-12){ \resizebox{!}{13.5cm}{\includegraphics[width=\paperwidth]{}}}; % Background image
      \draw[anchor=north] (midpoint) node [fill=deepblue!15!white,fill opacity=0.4,text opacity=1,inner sep=1cm]{\Huge\centering\bfseries\sffamily\parbox[c][][t]{\paperwidth}{\Huge\centering \textsf{Rapport de projet S5} \\[0pt] % Book title
      {\Large Département micro-ondes de Télécom-Bretagne}\\[20pt] 
      {\Huge Etude d’un filtre configurable basé sur l’utilisation de matériaux issus de la spintronic}}}; 
  \end{tikzpicture}};
\end{tikzpicture}

\title{}
\author{\bsc{Rodolphe} - \bsc{Jeunehomme}}

\vfill

\endgroup

%----------------------------------------------------------------------------------------
%	COPYRIGHT PAGE
%----------------------------------------------------------------------------------------

\newpage
~\vfill
\thispagestyle{empty}

\noindent \textsc{Rédigé par Rodolphe Jeunehomme}\\ % Publisher
\noindent \textsc{~\\Sous la direction de Vincent Castel}\\ % Publisher

\noindent \textit{Télécom-Bretagne, Novembre 2016 à Mars 2017} % Printing/edition date




%\cleardoublepage % Forces the first chapter to start on an odd page so it's on the right
\pagestyle{fancy} % Print headers again
\setcounter{part}{-1}
\part{Résumé - Executive summary}
\subsubsection{Français}

\subsubsection{English}


%----------------------------------------------------------------------------------------
%	TABLE OF CONTENTS
%----------------------------------------------------------------------------------------

\chapterimage{back8} % Table of contents heading image

\pagestyle{empty} % No headers

\tableofcontents % Print the table of contents itself

\pagestyle{fancy}
\part{Introduction}

\setcounter{chapter}{1}
\section{La technologie Spintronic}

\section{Le matériau magnétique Yttrium Iron Garnet}

\part{Le filtre configurable en fréquences}
Le filtre configurable est constitué d’un résonateur électrique et d’une
structure hybride à base d’un matériau magnétique isolant électriquement, l’Yttrium
Iron Garnet (YIG) et d’un métal normal, le platine (Pt). L’application d’un champ
magnétique statique sur le dispostif permettra de changer les propriétés résonantes du
filtre.
~\\\\La figure~\underline{\color{blue}\ref{filtre}} ci-dessous illustre le principe de fonctionnement du filtre configurable ainsi que ses différents composants.
\begin{figure}[H]
	\centering
	\itshape
	\includegraphics[width=15cm,height=9cm]{filtre_configurable.png}
	\caption{\label{filtre} \underline{Filtre configurable}}
\end{figure}
\noindent\footnotesize  \textbf{Figure~\underline{\color{blue}\ref{filtre}} | Filtre configurable. a}, Les paramètres S11 et \emph{S21} du résonateur électrique\footnote{ArXiv, \emph{Control of magnon-photon coupling strength in a planar resonator\slash YIG thin film configuration}} autour de 5 GHz. \textbf{b}, La courbe quadratique \footnote{LETTERS, Nature, \emph{Transmission of electrical signals by spin wave}} illustrant la dépendance magnétique H de résonance magnétique du YIG à une fréquence \emph{f0} donnée. \textbf{c}, Résultat du couplage entre les résonateurs magnétique et électrique\footnotemark[2]. La courbe \emph{\emph{S21}} en bleu illustre la fréquence Fo à laquelle le filtre résonne à champ magnétique nul. La courbe en rouge correspond au nouveau paramètre de transmission \emph{S21} du filtre à la condition de résonance magnétique du YIG. 
~\\\\
\normalsize Ce dispositif novateur sera à même de communiquer son état de fonctionnement par le
biais de phénomènes physiques complexes qui se traduisent en bout de chaîne par une simple tension
DC.~\\\\

\setcounter{chapter}{0}
\chapter{Etude des résonateurs \emph{Elliptika}} 
~\\\\\indent Plusieurs types de résonateurs électriques ont été conçus par l'entreprise brestoise \emph{Elliptika}, spécialisée dans la conception de circuits RF et hyperfréquences. Ces résonateurs électriques ont pour rôle de constituer la base du filtre configurable.
~\\\\La figure~\underline{\color{blue}\ref{resonateurs}} ci-dessous illustre différentes formes de résonateurs conçues par \emph{Elliptika}.
\begin{figure}[H]
	\centering
	\itshape
	\includegraphics[width=12cm,height=4cm]{resonateurs.png}
	\caption{\label{resonateurs} \underline{Résonateurs Elliptika}}
\end{figure}
\section{Les résonateurs de type \emph{Openloop}}
~\\\indent Plusieurs types de résonateurs \emph{Openloop} ont été réalisés. Un résonateur Openloop est caractérisé par son Gap (distance séparant les deux brins de la boucle). 
~\\\\A partir de résonateurs Openloop simples, différents modèles de résonateurs plus complexes et constitués de plusieurs résonateurs Openloop ont été réalisés. 
~\\
\subsubsection{Le résonateur Openloop res04\_3GHz}
\begin{figure}[H]
	\centering
	\itshape
	\includegraphics[width=3cm,height=4cm]{op_res04_3G.png}
	\caption{\label{op_res04_3G} \underline{Openloop res04\_3GHz}}
\end{figure}
\noindent 

~\\\\
\chapter{Etude du couplage de résonnance électrique et magnétique }
Le résonateur électrique utilisé pour les mesures est le résonateur à simple STUB présenté dans le chapitre précédent. Un matériau magnétique (YIG), sur lequel a été déposé par pulvérisation cathodique une fine couche de platine (Pt), est positionné sur le court-circuit du résonateur électrique et donc à la position du maximum de champs magnétique émis par le résonateur à simple STUB. Un champs magnétique statique est appliqué sur le dipositif afin de faire résonner magnétiquement le YIG et donc de pouvoir étudier le couplage des résonances électrique et magnétique.
\section{Le banc de mesures utilisé}
~\\\indent La photographie~\underline{\color{blue}\ref{banc}} ci-dessous illustre le banc de mesure utilisé. Ce banc de mesures est installé au département micro-ondes de Télécom-Bretagne.
\begin{figure}[H]
	\centering
	\itshape
	\includegraphics[width=12cm,height=7cm]{banc.png}
	\caption{\label{banc} \underline{Banc de mesures-1, laboratoire Spintronic}}
\end{figure}
\noindent\footnotesize  \textbf{Figure~\underline{\color{blue}\ref{banc}} | Banc de mesures-1. a}, un poste de travail \emph{Windows} et le logiciel \emph{LabView} afin de récolter automatiquement les données des différents instruments de mesures utilisés. \textbf{b}, un générateur de fréquences (9KHz-20GHz) \emph{KEYSIGHT}. \textbf{c}, un générateur de courant/tension \emph{KIKUSUI} pour générer le champs magnétique statique créé par les bobines (f). \textbf{d}, un Gaussmètre \emph{LakeShore} pour mesurer le champs magnétique satique créé par l'injection d'un courant dans les bobines (f). \textbf{e}, un générateur et détecteur de courant/tension \emph{KEITHLEY} pour générer un courant I dans le platine ou détecter une tension DC dans le platine. \textbf{f}, deux bobines servant à générer le champs magnétique statique et entre lesquelles est positionné le filtre. \textbf{g}, un analyseur de réseau pour l'étude des paramètres S du filtre.  
~\\\\
\normalsize La photographie~\underline{\color{blue}\ref{filtrebanc}} illustre le positionnement du filtre entre les deux bobines ainsi que le positionnement de l'association YIG/Pt sur le résonateur électrique:
\begin{figure}[H]
	\centering
	\itshape
	\includegraphics[width=12cm,height=7cm]{filtrebanc.png}
	\caption{\label{filtrebanc} \underline{Banc de mesures-2, laboratoire Spintronic}}
\end{figure}
\noindent\footnotesize  \textbf{Figure~\underline{\color{blue}\ref{filtrebanc}} | Banc de mesures-2. a}, le résonateur électrique à simple STUB. \textbf{b}, l'association YIG/PT à la position de court-circuit du résonateur électrique.
~\\\\
\normalsize Le filtre situé entre les deux bobines est le prototype de filtre configurable nous servant de test à l'étude du couplage de résonance électrique et magnétique. L'association YIG(\unit{6}{\micro\meter})/Pt(\unit{6}{\nano\meter}) est positionné sur le court-circuit du résonateur électrique à l'aide de laque d'argent afin de le fixer et de pouvoir conduire le courant entre les fils reliés au générateur de courant/tension \emph{KEITHLEY} et le platine.
\section{Mesure de la résonance électrique du filtre à simple STUB}
~\\\indent Le filtre à simple STUB utilisé est un filtre coupe-bande. La figure~\underline{\color{blue}\ref{fo}} ci-dessous réprésente les paramètres \emph{S11} de réflexion et \emph{S21} de transmission du filtre avec l'association YIG/Pt à champs magnétique nul.
\begin{figure}[H]
	\centering
	\itshape
	\includegraphics[width=10cm,height=7cm]{f0.png}
	\caption{\label{fo} \underline{Résonance électrique}}
\end{figure}
\noindent\footnotesize  \textbf{Figure~\underline{\color{blue}\ref{fo}} | Résonance électrique.} Le paramètre $ fo\simeq4.736\ GHz$ correspond à la fréquence à laquelle le filtre coupe-bande à simple STUB ne laisse plus passer le signal.
~\\\\
\normalsize Afin d'étudier la reproductibilité du dispositif et de déterminer une incertitude sur la valeur du paramètre de résonance \emph{f0}, plusieurs mesures du paramètres \emph{S21} du filtre avec l'association YIG/Pt ont été effectuées après avoir réinstallé plusieurs fois le dispositif entre les bobines. 
~\\La figure~\underline{\color{blue}\ref{foreprod}} ci-dessous, illustre l'incertitude de mesures sur la résonance électrique du dispositif.
\begin{figure}[H]
	\centering
	\itshape
	\includegraphics[width=10cm,height=7cm]{f0reprod.png}
	\caption{\label{foreprod} \underline{Résonance électrique - Repoductibilité des mesures}}
\end{figure}
\noindent L'incertitude sur la valeur du paramètre de transmission \emph{\emph{S21}} est plus élevé lorsque celle-ci est proche de 0. L'incertitude trouvée à hauteur de la fréquence \emph{f0} est $ fo\simeq4.740\pm0.015\ GHz$.
\section{La résonance magnétique de l'Yttrium Iron Garnet}
~\\\indent L'Yttrium Iron Garnet ou YIG entre en résonance magnétique sous l'effet d'un de l'application d'un champs magnétique H statique. 
~\\La figure~\underline{\color{blue}\ref{resomag}} ci-dessous illustre la courbe théorique de résonance magnétique du YIG\footnote{ArXiv, \emph{Control of magnon-photon coupling strength in a planar resonator/YIG thin film configura-
tion}} en fonction du champs magnétique qui lui est appliqué.
\begin{figure}[H]
	\centering
	\itshape
	\includegraphics[width=6cm,height=4cm]{resomag.png}
	\caption{\label{resomag} \underline{Résonance magnétique du YIG}}
\end{figure}
\noindent La courbe de résonance magnétique du YIG est d'allure quadratique et permet de déterminer la valeur \emph{Hres} du champs magnétique à appliquer à la fréquence de résonance électrique \emph{f0} du filtre pour faire résonner le YIG. 
\section{Le couplage des résonances électrique et magnétique}
\subsection{Mise en évidence du couplage de résonance}
\noindent Afin de mettre en évidence le couplage de résonance, un balayage en champs magnétique avec un pas arbitraire à la fréquence \emph{f0} de résonance électrique du filtre (résonateur électrique + association YIG/PT) a été effectué.
~\\La figure~\underline{\color{blue}\ref{Hres}} ci-dessous illustre l'allure du paramètre de transmission \emph{S21} du filtre à \emph{H=983.48 Oe}, \emph{Hres=995.92 Oe} et \emph{H=1010.58 Oe}.
\begin{figure}[H]
	\centering
	\itshape
	\includegraphics[width=15cm,height=5cm]{H.png}
	\caption{\label{Hres} \underline{Paramètre S21 autour de H = Hres}}
\end{figure}
\noindent Lorsqu'un champs magnétique \emph{H} statique est appliqué sur le dispositif \emph{filtre + association YIG/PT}, un phénomène de séparation de la fréquence de résonance \emph{f0} en deux fréquences distinctes \emph{f1} et \emph{f2} apparaît. Ce phénomène de \emph{splitting} est dû au couplage des résonances électrique et magnétique.
La présence d'irrégularités le long des pentes des courbes des pics de résonance est dûe à la résonance des autres modes magnétique présent dans le \emph{YIG} dont le nombre dépend des dimensions de ce dernier\footnote{Seul la résonance du mode fondamental est étudiée dans ce rapport.}. 
\subsection{Balayage en champs et en fréquences}
\noindent Afin d'étudier le comportement de ce phénomène de \emph{splitting}, un balayage en champs et en fréquence avec respectivement un pas de \emph{2.9 Oe} et de \emph{210 KHz} a été effectué. Les données stockées par le logiciel \emph{Labview} ont été analysées à l'aide d'un script \emph{matlab}.
~\\La figure~\underline{\color{blue}\ref{matH}} ci-dessous représente la dépendance en champs et en fréquence du paramètre de transmission \emph{S21}.
\begin{figure}[H]
	\centering
	\itshape
	\includegraphics[width=15cm,height=10cm]{matH.png}
	\caption{\label{matH} \underline{Paramètre S21 - Balayage en champs magnétique et en fréquences}}
\end{figure}
\noindent Le paramètre S21 est ici représenté par un code couleur. Il varie du rouge foncé pour 0 dB au bleu foncé pour -12 dB (c'est à dire lorsque le signal ne passe plus). Les conditions de résonance maximum du filtre correspondent aux coordonnées des zones du graphe ou la couleur est proche du bleu foncé. 
On observe ainsi l'apparition d'une bande de résonance interdite.
\subsection{Utilisation du modèle de couplage harmonique}
\noindent Le modèle de couplage harmonique\footnote{Arxiv, \emph{Study of the cavity-magnon-polariton transmission line shape}.} entre les résonances électrique et magnétique est utilisé dans le but de déterminer les fomules analytiques qui permettront de décrire le comportement de la variation des fréquences de résonances \emph{f1} et \emph{f2} en fonction du champs magnétique statique appliqué sur le filtre. 
~\\Ce modèle s'appuie sur les formules analytiques suivantes :
$$\underset{1}{f}\ \text{=}\ \frac{1}{2}*(\underset{0}{f}\text{+}\underset{r}{f})\text{+}\sqrt{(\underset{0}{f}-\underset{r}{f})^{2}\text{+}k^4*\underset{0}{f}^2}$$
$$\underset{2}{f}\ \text{=}\ \frac{1}{2}*(\underset{0}{f}\text{+}\underset{r}{f})-\sqrt{(\underset{0}{f}-\underset{r}{f})^{2}\text{+}k^4*\underset{0}{f}^2}$$
avec dans le cas du YIG,
$$\underset{r}{f}\ \text{=}\ \frac{1.8*10^7}{2*\pi}*\sqrt{|H*(H\text{+}\underset{s}{M})|}$$
et,
$$k\ \text{=}\ \sqrt{\frac{\underset{gap}{f}}{\underset{0}{f}}}\ ,\ \ \underset{gap}{f}\ \text{=}\ \underset{2}{f}-\underset{1}{f}\ \text{\emph{à H=Hres,}}\ \ \underset{s}{M}\equiv\text{\emph{Aimentation du YIG à saturation.}}$$
~\\En jouant sur la valeur du paramètre \emph{k}, associé aux pertes magnétiques dans le YIG, et celle de \emph{Ms}, les courbes représentatives des formules du modèle de couplage harmonique
de \emph{f1} et \emph{f2} se superposent parfaitement avec celle des positions des minimums de \emph{S21}, c'est à dire les points où se situe la résonance. Les courbes se superposent lorsque $k\text{=}0.1504$ et $Ms\text{=}1750 A.m^{-1}$.
~\\Les courbes de \emph{f1} et \emph{f2} avec les valeurs de \emph{k} et \emph{Ms} ainsi trouvées sont représentées en rose sur la figure~\underline{\color{blue}\ref{matH}}.
La courbe noire horizontale sur cette même figure représente la résonance \emph{f0} du filtre lorsque $H\text{=}0\ Oe$. Les fréquences \emph{f1} et \emph{f2} tendent toutes deux vers \emph{f0} lorsque le champs magnétique tend vers $Hres\pm\infty$.  
~\\\\Ceci démontre que le modèle de couplage harmonique est un bon modèle pour représenter analytiquement le couplage de résonance électrique et magnétique dans le filtre.
\subsection{Détermination du champs magnétique résonant}
\noindent Toujours en se basant sur le résultat des mesures en champs et en fréquences du paramètre de transmission \emph{S21} (voir figure~\underline{\color{blue}\ref{matH}}), il est possible de déterminer ainsi la valeur exacte \emph{Hres} du champs magnétique à appliquer sur le filtre tel que le couplage de résonance électrique et magnétique soit au maximum.
~\\En effet, d'après le modèle de couplage harmonique\footnotemark[3], le couplage entre les résonances électrique et magnétique est maximum lorsque $\underset{0}{f}-\underset{1}{f}\ \text{=}\ \underset{2}{f}-\underset{0}{f}$.
~\\La figure~\underline{\color{blue}\ref{fifo}} ci-dessous représente les courbes $\underset{0}{f}-\underset{1}{f}$ en bleu et $\underset{2}{f}-\underset{0}{f}$ en rouge:
\begin{figure}[H]
	\centering
	\itshape
	\includegraphics[width=10cm,height=7cm]{fifo.png}
	\caption{\label{fifo} \underline{Distance des fréquences de résonance f1 et f2 à f0}}
\end{figure}
\noindent Les courbes se croisent au point d'abscisse $ Hres\simeq995.92\ Oe$. Lorsque le champs magnétique appliqué est voisin de \emph{Hres}, l'écart des fréquences de résonannces \emph{f1} et \emph{f2} avec la fréquence fondamentale \emph{f0} se comporte de manière non-linéaire en fonction du champs magnétique appliqué.
De plus, la position du point d'abscisse \emph{Hres} correspond à un \emph{gap} de $ \underset{gap}{f}\simeq634\ MHz$. Le \emph{gap} à \emph{H=Hres} à une valeur plus proche de celle du minimum de $\underset{2}{f}-\underset{1}{f}$ que de celle de son maximum. 
~\\Le segment noir et verticale \emph{AB} sur la figure~\underline{\color{blue}\ref{matH}} illustre la position du \emph{gap} à \emph{H=Hres} ainsi que son écartement.
~\\\\La figure~\underline{\color{blue}\ref{hres_cf}} suivante illuste la valeur du paramètre \emph{S21} en dB à \emph{H=0 Oe} et à \emph{H=Hres}:
\begin{figure}[H]
	\centering
	\itshape
	\includegraphics[width=15cm,height=8cm]{hres_cf.png}
	\caption{\label{hres_cf} \underline{Hybridation de la résonance à H=Hres}}
\end{figure}
\noindent Pour une valeur de champs appliqué \emph{H=Hres}, il y a donc bien une hybridation de la résonance \emph{f0} du mode fondamentale en deux résonances distinctes, \emph{f1} et \emph{f2}. Cette hybridation résulte du couplage entre les résonances électrique du filtre à simple STUB et magnétique du YIG.
\subsection{Balayage en courant et en fréquences}
\noindent Afin d'établir un contrôle sur les pertes magnétiques du \emph{YIG} et de pouvoir évaluer la configurabilité de la résonance du filtre, un courant I est injecté dans le platine. Ce courant injecté va être converti en onde de spin dans le \emph{YIG} grâce à la combinaison des phénomènes de \emph{Spin Hall Effect} et de \emph{Spin Torque Transfer}\footnote{LETTERS Nature, \emph{Transmission of electrical signals by spin wave.}} et modifier les propriétés résonantes du filtre.
Un champs magnétique \emph{H=Hres} est appliqué statiquement sur le filtre.
~\\ A l'aide du logiciel \emph{Labview}, un balayage en courant avec un pas de \emph{2 mA} et en fréquences avec un pas de \emph{210 KHz} est effectué. Les résultats de mesures sur la valeur du paramètre \emph{S21} sont illustrés sur la figure~\underline{\color{blue}\ref{isweep}} suivante:
\begin{figure}[H]
	\centering
	\itshape
	\includegraphics[width=15cm,height=10cm]{isweep.png}
	\caption{\label{isweep} \underline{Paramètre S21 - Balayage en courant et en fréquences}}
\end{figure}

\part{Conclusions}

\setcounter{part}{-5}
\part{Annexes}

\setcounter{part}{-6}
\part{Bibliographie}
\label{Sources}
\appendix
\begin{description}
 \item[$\bullet$] ArXiv, \emph{Control of magnon-photon coupling strength in a planar resonator/YIG thin film configuration}
 \item[$\bullet$] LETTERS, Nature, \emph{Transmission of electrical signals by spin wave}
 \item[$\bullet$] Arxiv, \emph{Study of the cavity-magnon-polariton transmission line shape}.
\end{description}

\setcounter{part}{-7}
\part{Glossaire}
\begin{description}
 \item[$\bullet$] YIG: \emph{Yittrium Iron Garnet}
 \item[$\bullet$] Pt: \emph{Platine}
\end{description}


\end{document}
