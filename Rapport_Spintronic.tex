%----------------------------------------------------------------------------------------
%	PACKAGES AND OTHER DOCUMENT CONFIGURATIONS
%----------------------------------------------------------------------------------------
\nonstopmode
\documentclass[12pt,fleqn]{book} % Default font size and left-justified equations
\renewcommand*{\rmdefault}{Roboto}
%----------------------------------------------------------------------------------------
\input{structure} % Insert the commands.tex file which contains the majority of the structure behind the template
%\pagestyle{fancy}

\begin{document}	

%----------------------------------------------------------------------------------------
%	TITLE PAGE
%----------------------------------------------------------------------------------------
~\\
\begin{figure}

\end{figure}

\begingroup
\thispagestyle{empty}
\begin{tikzpicture}[remember picture,overlay]
  \coordinate [below=12cm] (midpoint) at (current page.north);
  \node at (current page.north west)
  {\begin{tikzpicture}[remember picture,overlay]
      %\node[anchor=north west,inner sep=0pt] at (0,-12){ \resizebox{!}{13.5cm}{\includegraphics[width=\paperwidth]{}}}; % Background image
      \draw[anchor=north] (midpoint) node [fill=deepblue!15!white,fill opacity=0.4,text opacity=1,inner sep=1cm]{\Huge\centering\bfseries\sffamily\parbox[c][][t]{\paperwidth}{\Huge\centering \textsf{Rapport de projet S5} \\[0pt] % Book title
      {\Large Département micro-ondes de Télécom-Bretagne}\\[20pt] 
      {\Huge Etude d’un filtre configurable basé sur l’utilisation de matériaux issus de la spintronic}}}; 
  \end{tikzpicture}};
\end{tikzpicture}

\title{}
\author{\bsc{Rodolphe} - \bsc{Jeunehomme}}

\vfill

\endgroup

%----------------------------------------------------------------------------------------
%	COPYRIGHT PAGE
%----------------------------------------------------------------------------------------

\newpage
~\vfill
\thispagestyle{empty}

\noindent \textsc{Rédigé par Rodolphe Jeunehomme}\\ % Publisher
\noindent \textsc{~\\Sous la direction de Vincent Castel}\\ % Publisher

\noindent \textit{Télécom-Bretagne, Novembre 2016 à Mars 2017} % Printing/edition date




%\cleardoublepage % Forces the first chapter to start on an odd page so it's on the right
\pagestyle{fancy} % Print headers again
\setcounter{part}{-1}
\part{Résumé - Executive summary}
\subsubsection{Français}

\subsubsection{English}


%----------------------------------------------------------------------------------------
%	TABLE OF CONTENTS
%----------------------------------------------------------------------------------------

\chapterimage{back8} % Table of contents heading image

\pagestyle{empty} % No headers

\tableofcontents % Print the table of contents itself

\pagestyle{fancy}
\part{Introduction}
Le département micro-ondes de Télécom-Bretagne souhaite concevoir un filtre à
largeur de bande configurable. Ce filtre est une application des concepts de spintronique et d’objets communicants.
~\\\\En 2010, une équipe de recherche japonaise a publié un article\footnote{LETTERS, Nature, \emph{Transmission of electrical signals by spin wave}} détaillant la transmission d’un signal électrique dans un isolant. Ce nouveau concept de transmission de
signal est une application du concept de spintronique dans les matériaux magnétiques
isolant électriquement. De nombreux travaux de recherches ont émergé suite à la mise
en évidence de cette preuve de concept et ont permis de faire progresser l’état de l’art.
~\\\\Dans le cadre de ce projet, le filtre configurable sera constitué d’un résonateur et d’une
structure hybride à base d’un matériau magnétique isolant électriquement, l’Yttrium
Iron Garnet (YIG) et d’un métal normal, le platine (Pt). L’application d’un champ
magnétique statique sur le dispostif permettra de changer les propriétés résonantes du
filtre.
~\\\\Les objectifs de ce projet sont:
~\\
\begin{itemize}
 \item[$\bullet$] La conception d'un prototype de filtre configurable fonctionnant à l'air libre et à température ambiante (la conception d'un filtre à largeur de bande configurable a déjà été étudiée par une équipe scientifique en 2015\footnote{Physical Review Letters, \emph{Spin Pumping In Electrodynamically Coupled Magnon-Photon Systems}} au sein d'une cavité résonnante.
 \item[$\bullet$] L'étude de l'intéraction magnon-photon en utilisant l'association YIG/Pt (couplage des résonances électrique du filtre et magnétique du YIG).
 \item[$\bullet$] Le contrôle des pertes magnétiques dans le YIG avec la mise en jeu des phénomènes de \emph{Spin Hall Effect} (SHE) et de \emph{Spin Transfer Torque} (STT).
 \item[$\bullet$] La détection d’un courant de spin grâce aux phénomènes physiques réciproques au SHE et au STT que sont les phénomènes de \emph{Spin Pumping} (SP) et de \emph{Inverse Spin Hall Effect} (ISHE). La présence d'un tel courant sera mis en évidence par la détection d'une tension DC aux bornes du platine partageant une interface commune avec le YIG.
 \item[$\bullet$] La conception d’un boîtier sous le logiciel ADS permettant d’obtenir une structure fiable et solide qui servira à contenir au sein d’une cavité le matériau magnétique (YIG) sur lequel
une couche de platine (Pt) d’épaisseur nanométrique a été déposée par pulvérisation
cathodique.
\end{itemize}
\setcounter{chapter}{1}
\subsubsection{La technologie Spintronic}
\noindent Cette thématique de recherche, communément appelé spintronic ou électronique de
spin, fut révélée au grand public lors de l’obtention du prix Nobel de physique par Albert
Fert en 2007 pour ses travaux sur la magnétorésistance géante. Cette découverte
a joué un rôle cruciale sur l’augmentation drastique de la capacité de nos disques dur
magnétique (de 1Gb/in² à 600 Gb/in² avec l'utilisation des têtes GMRs).
~\\\\
\begin{figure}[H]
	\centering
	\itshape
	\includegraphics[width=15cm,height=9cm]{introspin.png}
	\caption{\label{introspin} \underline{Spintronic}}
\end{figure}
\subsubsection{Le matériau magnétique Yttrium Iron Garnet}
\noindent L'Yttrium Iron Garnet ou YIG est le matériau magnétique le plus utilisé dans les systèmes radiofréquences. C'est un matériau magnétique isolant électriquement mais aussi celui possédant le moins de pertes magnétiques (inférieures à $10^{-4}$).
\begin{figure}[H]
	\centering
	\itshape
	\includegraphics[width=8cm,height=3cm]{introyig.png}
	\caption{\label{introspin} \underline{Yttrium Iron Garnet}}
\end{figure}
\part{Le filtre configurable en fréquences}
Le filtre configurable est constitué d’un résonateur électrique et d’une
structure hybride à base d’un matériau magnétique isolant électriquement, l’Yttrium
Iron Garnet (YIG) et d’un métal normal, le platine (Pt). L’application d’un champ
magnétique statique sur le dispostif permettra de changer les propriétés résonantes du
filtre.
~\\\\La figure~\underline{\color{blue}\ref{filtre}} ci-dessous illustre le principe de fonctionnement théorique du filtre configurable ainsi que ses différents composants.
\begin{figure}[H]
	\centering
	\itshape
	\includegraphics[width=15cm,height=9cm]{filtre_configurable.png}
	\caption{\label{filtre} \underline{Filtre configurable}}
\end{figure}
\noindent\footnotesize  \textbf{Figure~\underline{\color{blue}\ref{filtre}} | Filtre configurable. a}, Les paramètres \emph{S11} et \emph{S21} du résonateur électrique\footnote{ArXiv, \emph{Control of magnon-photon coupling strength in a planar resonator\slash YIG thin film configuration}} autour de 5 GHz. \textbf{b}, La courbe quadratique \footnote{LETTERS, Nature, \emph{Transmission of electrical signals by spin wave}} illustrant la dépendance magnétique H de résonance magnétique du YIG à une fréquence \emph{f0} donnée. \textbf{c}, Résultat du couplage entre les résonateurs magnétique et électrique\footnotemark[2]. La courbe \emph{S21} en bleu illustre la fréquence \emph{f0} à laquelle le filtre résonne à champ magnétique nul. La courbe en rouge correspond au nouveau paramètre de transmission \emph{S21} du filtre à la condition de résonance magnétique du YIG. 
~\\\\
\normalsize Ce dispositif novateur sera à même de communiquer son état de fonctionnement par le
biais de phénomènes physiques complexes qui se traduisent en bout de chaîne par une simple tension
DC.
~\\\\Le but de cette partie est d'étudier par la mesure le \emph{couplage des résonances magnétique et électrique} dans le filtre afin de pouvoir identifier les modèles théoriques physiques permettant de pouvoir contrôler et prévoir la configurabilité du filtre en fonction d'un champs magnétique statique appliqué ou d'un courant injecté dans le platine.

\setcounter{chapter}{0}
\chapter{Etude des résonateurs \emph{Elliptika}} 
~\\\\\indent Plusieurs types de résonateurs électriques ont été conçus par l'entreprise brestoise \emph{Elliptika}, spécialisée dans la conception de circuits RF et hyperfréquences. Ces résonateurs électriques ont pour rôle de constituer la base du filtre configurable.
~\\\\La figure~\underline{\color{blue}\ref{resonateurs}} ci-dessous illustre différentes formes de résonateurs conçues par \emph{Elliptika}.
\begin{figure}[H]
	\centering
	\itshape
	\includegraphics[width=12cm,height=4cm]{resonateurs.png}
	\caption{\label{resonateurs} \underline{Résonateurs Elliptika}}
\end{figure}
\section{Les résonateurs de type \emph{Openloop}}
~\\\noindent Plusieurs types de résonateurs \emph{Openloop} ont été réalisés. Un résonateur Openloop est caractérisé par son Gap (distance séparant les deux brins de la boucle). 
~\\\\A partir de résonateurs Openloop simples, différents modèles de résonateurs plus complexes et constitués de plusieurs résonateurs Openloop ont été réalisés. 
~\\
\subsubsection{Le résonateur Openloop res04\_3GHz}
\begin{figure}[H]
	\centering
	\itshape
	\includegraphics[width=3cm,height=4cm]{op_res04_3G.png}
	\caption{\label{op_res04_3G} \underline{Openloop res04\_3GHz}}
\end{figure}
\noindent 

~\\\\
\chapter{Etude du couplage de résonnance électrique et magnétique }
Le résonateur électrique utilisé pour les mesures est le résonateur à simple STUB présenté dans le chapitre précédent. Un matériau magnétique (YIG), sur lequel a été déposé par pulvérisation cathodique une fine couche de platine (Pt), est positionné sur le court-circuit du résonateur électrique et donc à la position du maximum de champs magnétique émis par le résonateur à simple STUB. Un champs magnétique statique est appliqué sur le dipositif afin de faire résonner magnétiquement le YIG et donc de pouvoir étudier le couplage des résonances électrique et magnétique.
\section{Le banc de mesures utilisé}
~\\\noindent La photographie~\underline{\color{blue}\ref{banc}} ci-dessous illustre le banc de mesure utilisé. Ce banc de mesures est installé au département micro-ondes de Télécom-Bretagne.
\begin{figure}[H]
	\centering
	\itshape
	\includegraphics[width=12cm,height=7cm]{banc.png}
	\caption{\label{banc} \underline{Banc de mesures-1, laboratoire Spintronic}}
\end{figure}
\noindent\footnotesize  \textbf{Figure~\underline{\color{blue}\ref{banc}} | Banc de mesures-1. a}, un poste de travail \emph{Windows} et le logiciel \emph{LabView} afin de récolter automatiquement les données des différents instruments de mesures utilisés. \textbf{b}, un générateur de fréquences (9KHz-20GHz) \emph{KEYSIGHT}. \textbf{c}, un générateur de courant/tension \emph{KIKUSUI} pour générer le champs magnétique statique créé par les bobines (f). \textbf{d}, un Gaussmètre \emph{LakeShore} pour mesurer le champs magnétique satique créé par l'injection d'un courant dans les bobines (f). \textbf{e}, un générateur et détecteur de courant/tension \emph{KEITHLEY} pour générer un courant I dans le platine ou détecter une tension DC dans le platine. \textbf{f}, deux bobines servant à générer le champs magnétique statique et entre lesquelles est positionné le filtre. \textbf{g}, un analyseur de réseau pour l'étude des paramètres S du filtre.  
~\\\\
\normalsize La photographie~\underline{\color{blue}\ref{filtrebanc}} illustre le positionnement du filtre entre les deux bobines ainsi que le positionnement de l'association YIG/Pt sur le résonateur électrique:
\begin{figure}[H]
	\centering
	\itshape
	\includegraphics[width=12cm,height=7cm]{filtrebanc.png}
	\caption{\label{filtrebanc} \underline{Banc de mesures-2, laboratoire Spintronic}}
\end{figure}
\noindent\footnotesize  \textbf{Figure~\underline{\color{blue}\ref{filtrebanc}} | Banc de mesures-2. a}, le résonateur électrique à simple STUB. \textbf{b}, l'association YIG/PT à la position de court-circuit du résonateur électrique.
~\\\\
\normalsize Le filtre situé entre les deux bobines est le prototype de filtre configurable nous servant de test à l'étude du couplage de résonance électrique et magnétique. L'association YIG(\unit{6}{\micro\meter})/Pt(\unit{6}{\nano\meter}) est positionné sur le court-circuit du résonateur électrique à l'aide de laque d'argent afin de le fixer et de pouvoir conduire le courant entre les fils reliés au générateur de courant/tension \emph{KEITHLEY} et le platine.
\section{Mesure de la résonance électrique du filtre à simple STUB}
~\\\noindent Le filtre à simple STUB utilisé est un filtre coupe-bande. La figure~\underline{\color{blue}\ref{fo}} ci-dessous réprésente les paramètres \emph{S11} de réflexion et \emph{S21} de transmission du filtre avec l'association YIG/Pt à champs magnétique nul.
\begin{figure}[H]
	\centering
	\itshape
	\includegraphics[width=10cm,height=7cm]{f0.png}
	\caption{\label{fo} \underline{Résonance électrique}}
\end{figure}
\noindent\footnotesize  \textbf{Figure~\underline{\color{blue}\ref{fo}} | Résonance électrique.} Le paramètre $ fo\simeq4.736\ GHz$ correspond à la fréquence à laquelle le filtre coupe-bande à simple STUB ne laisse plus passer le signal.
~\\\\
\normalsize Afin d'étudier la reproductibilité du dispositif et de déterminer une incertitude sur la valeur du paramètre de résonance \emph{f0}, plusieurs mesures du paramètres \emph{S21} du filtre avec l'association YIG/Pt ont été effectuées après avoir réinstallé plusieurs fois le dispositif entre les bobines. 
~\\La figure~\underline{\color{blue}\ref{foreprod}} ci-dessous, illustre l'incertitude de mesures sur la résonance électrique du dispositif.
\begin{figure}[H]
	\centering
	\itshape
	\includegraphics[width=10cm,height=7cm]{f0reprod.png}
	\caption{\label{foreprod} \underline{Résonance électrique - Repoductibilité des mesures}}
\end{figure}
\noindent L'incertitude sur la valeur du paramètre de transmission \emph{\emph{S21}} est plus élevé lorsque celle-ci est proche de 0. L'incertitude trouvée à hauteur de la fréquence \emph{f0} est $ fo\simeq4.740\pm0.015\ GHz$.
\section{La résonance magnétique de l'Yttrium Iron Garnet}
~\\\noindent L'Yttrium Iron Garnet ou YIG entre en résonance magnétique sous l'effet d'un de l'application d'un champs magnétique H statique. 
~\\La figure~\underline{\color{blue}\ref{resomag}} ci-dessous illustre la courbe théorique de résonance magnétique du YIG\footnote{ArXiv, \emph{Control of magnon-photon coupling strength in a planar resonator/YIG thin film configura-
tion}} en fonction du champs magnétique qui lui est appliqué.
\begin{figure}[H]
	\centering
	\itshape
	\includegraphics[width=6cm,height=4cm]{resomag.png}
	\caption{\label{resomag} \underline{Résonance magnétique du YIG}}
\end{figure}
\noindent La courbe de résonance magnétique du YIG est d'allure quadratique et permet de déterminer la valeur \emph{Hres} du champs magnétique à appliquer à la fréquence de résonance électrique \emph{f0} du filtre pour faire résonner le YIG. 
\section{Le couplage des résonances électrique et magnétique}
\subsection{Mise en évidence du couplage de résonance}
\noindent Afin de mettre en évidence le couplage de résonance, un balayage en champs magnétique avec un pas arbitraire à la fréquence \emph{f0} de résonance électrique du filtre (résonateur électrique + association YIG/PT) a été effectué.
~\\La figure~\underline{\color{blue}\ref{Hres}} ci-dessous illustre l'allure du paramètre de transmission \emph{S21} du filtre à \emph{H=983.48 Oe}, \emph{Hres=995.92 Oe} et \emph{H=1010.58 Oe}.
\begin{figure}[H]
	\centering
	\itshape
	\includegraphics[width=15cm,height=5cm]{H.png}
	\caption{\label{Hres} \underline{Paramètre S21 autour de H = Hres}}
\end{figure}
\noindent Lorsqu'un champs magnétique \emph{H} statique est appliqué sur le dispositif \emph{filtre + association YIG/PT}, un phénomène de séparation de la fréquence de résonance \emph{f0} en deux fréquences distinctes \emph{f1} et \emph{f2} apparaît. Ce phénomène de \emph{splitting} est dû au couplage des résonances électrique et magnétique.
La présence d'irrégularités le long des pentes des courbes des pics de résonance est dûe à la résonance des autres modes magnétique présent dans le \emph{YIG} dont le nombre dépend des dimensions de ce dernier\footnote{Seul la résonance du mode fondamental est étudiée dans ce rapport.}. 
\subsection{Balayage en champs et en fréquences}
\noindent Afin d'étudier le comportement de ce phénomène de \emph{splitting}, un balayage en champs et en fréquence avec respectivement un pas de \emph{2.9 Oe} et de \emph{210 KHz} a été effectué. Les données stockées par le logiciel \emph{Labview} ont été analysées à l'aide d'un script \emph{matlab}.
~\\La figure~\underline{\color{blue}\ref{matH}} ci-dessous représente la dépendance en champs et en fréquence du paramètre de transmission \emph{S21}.
\begin{figure}[H]
	\centering
	\itshape
	\includegraphics[width=15cm,height=10cm]{matH.png}
	\caption{\label{matH} \underline{Paramètre S21 - Balayage en champs magnétique et en fréquences}}
\end{figure}
\noindent\footnotesize\textbf{Figure~\underline{\color{blue}\ref{matH}} | Balayage en champs et en fréquences. Courbes roses}, les courbes des minimums locaux du paramètre de transmission \emph{S21} aux alentours des fréquences \emph{f1} et \emph{f2} auxquelles se superposent les courbes de \emph{f1} et \emph{f2} du molède de couplage harmonique pour une valeur de coefficient de couplage $k\ \text{=}\ 0.1504$ et d'aimentation du YIG à saturation $Ms\ \text{=}\ 1750\ A.m^{-1}$. \textbf{Segment AB}, la valeur du \emph{gap} de fréquences pour une valeur de champs injecté dans le platine $H\ \text{=}\ Hres\ \text{=}\ 995.92\ 0e$. \textbf{Courbe noire horizontale}, la fréquence de résonance \emph{f0} du filtre à champs nul.
~\\\\
\normalsize
\noindent Le paramètre S21 est ici représenté par un code couleur. Il varie du rouge foncé pour 0 dB au bleu foncé pour -12 dB (c'est à dire lorsque le signal ne passe plus). Les conditions de résonance maximum du filtre correspondent aux coordonnées des zones du graphe ou la couleur est proche du bleu foncé.
~\\La position en fréquence des résonances \emph{f1} et \emph{f2} varie en fonction de la valeur du champs magnétique injecté dans le YIG. On observe ainsi l'apparition d'une bande de résonance interdite.
\subsection{Utilisation du modèle de couplage harmonique}
\noindent Le modèle de couplage harmonique\footnote{Arxiv, \emph{Study of the cavity-magnon-polariton transmission line shape}.} entre les résonances électrique et magnétique est utilisé dans le but de déterminer les fomules analytiques qui permettront de décrire le comportement de la variation des fréquences de résonances \emph{f1} et \emph{f2} en fonction du champs magnétique statique appliqué sur le filtre. 
~\\Ce modèle s'appuie sur les formules analytiques suivantes :
$$\underset{1}{f}\ \text{=}\ \frac{1}{2}*(\underset{0}{f}\text{+}\underset{r}{f})\text{+}\sqrt{(\underset{0}{f}-\underset{r}{f})^{2}\text{+}k^4*\underset{0}{f}^2}$$
$$\underset{2}{f}\ \text{=}\ \frac{1}{2}*(\underset{0}{f}\text{+}\underset{r}{f})-\sqrt{(\underset{0}{f}-\underset{r}{f})^{2}\text{+}k^4*\underset{0}{f}^2}$$
avec dans le cas du YIG,
$$\underset{r}{f}\ \text{=}\ \frac{1.8*10^7}{2*\pi}*\sqrt{|H*(H\text{+}\underset{s}{M})|}$$
et,
$$k\ \text{=}\ \sqrt{\frac{\underset{gap}{f}}{\underset{0}{f}}}\ ,\ \ \underset{gap}{f}\ \text{=}\ \underset{2}{f}-\underset{1}{f}\ \text{\emph{à H=Hres,}}\ \ \underset{s}{M}\equiv\text{\emph{Aimentation du YIG à saturation.}}$$
~\\En jouant sur la valeur du paramètre \emph{k}, associé aux pertes magnétiques dans le YIG, et celle de \emph{Ms}, les courbes représentatives des formules du modèle de couplage harmonique
de \emph{f1} et \emph{f2} se superposent parfaitement avec celle des positions des minimums de \emph{S21}, c'est à dire les points où se situe la résonance. Les courbes se superposent lorsque $k\text{=}0.1504$ et $Ms\text{=}1750 A.m^{-1}$.
~\\Les courbes de \emph{f1} et \emph{f2} avec les valeurs de \emph{k} et \emph{Ms} ainsi trouvées sont représentées en rose sur la figure~\underline{\color{blue}\ref{matH}}.
La courbe noire horizontale sur cette même figure représente la résonance \emph{f0} du filtre lorsque $H\text{=}0\ Oe$. Les fréquences \emph{f1} et \emph{f2} tendent toutes deux vers \emph{f0} lorsque le champs magnétique tend vers $Hres\pm\infty$.  
~\\\\Ceci démontre que le modèle de couplage harmonique est un bon modèle pour représenter analytiquement le couplage de résonance électrique et magnétique dans le filtre.
\subsection{Détermination du champs magnétique résonant}
\noindent Toujours en se basant sur le résultat des mesures en champs et en fréquences du paramètre de transmission \emph{S21} (voir figure~\underline{\color{blue}\ref{matH}}), il est possible de déterminer ainsi la valeur exacte \emph{Hres} du champs magnétique à appliquer sur le filtre tel que le couplage de résonance électrique et magnétique soit au maximum.
~\\En effet, d'après le modèle de couplage harmonique\footnotemark[3], le couplage entre les résonances électrique et magnétique est maximum lorsque $\underset{0}{f}-\underset{1}{f}\ \text{=}\ \underset{2}{f}-\underset{0}{f}$.
~\\La figure~\underline{\color{blue}\ref{fifo}} ci-dessous représente les courbes $\underset{0}{f}-\underset{1}{f}$ en bleu et $\underset{2}{f}-\underset{0}{f}$ en rouge:
\begin{figure}[H]
	\centering
	\itshape
	\includegraphics[width=10cm,height=7cm]{fifo.png}
	\caption{\label{fifo} \underline{Distance des fréquences de résonance f1 et f2 à f0}}
\end{figure}
\noindent Les courbes se croisent au point d'abscisse $ Hres\simeq995.92\ Oe$. Lorsque le champs magnétique appliqué est voisin de \emph{Hres}, l'écart des fréquences de résonannces \emph{f1} et \emph{f2} avec la fréquence fondamentale \emph{f0} se comporte de manière non-linéaire en fonction du champs magnétique appliqué.
De plus, la position du point d'abscisse \emph{Hres} correspond à un \emph{gap} de $ \underset{gap}{f}\simeq634\ MHz$. Le \emph{gap} à \emph{H=Hres} à une valeur plus proche de celle du minimum de $\underset{2}{f}-\underset{1}{f}$ que de celle de son maximum. 
~\\Le segment noir et verticale \emph{AB} sur la figure~\underline{\color{blue}\ref{matH}} illustre la position du \emph{gap} à \emph{H=Hres} ainsi que son écartement.
~\\\\La figure~\underline{\color{blue}\ref{hres_cf}} suivante illuste la valeur du paramètre \emph{S21} mesuré en dB à \emph{H=0 Oe} et à \emph{H=Hres}:
\begin{figure}[H]
	\centering
	\itshape
	\includegraphics[width=15cm,height=8cm]{hres_cf.png}
	\caption{\label{hres_cf} \underline{Hybridation de la résonance à H=Hres}}
\end{figure}
\noindent Pour une valeur de champs appliqué \emph{H=Hres}, il y a donc bien une hybridation de la résonance \emph{f0} du mode fondamentale en deux résonances distinctes, \emph{f1} et \emph{f2}. Cette hybridation résulte du couplage entre les résonances électrique du filtre à simple STUB et magnétique du YIG.
~\\\\ Le positionnement de la résonance de plus haute fréquence \emph{f2} est précis à quelques MHz (<10). En revanche, celui de la résonance \emph{f1} est précis à environ 20 Mhz. L'incertitude sur le résultat de la position de la résonance \emph{f1} est donc plus grande commme on peut le constater sur l'allure du paramètre S21 au niveau de la résonance \emph{f1} à \emph{H=Hres}.
~\\\\ A \emph{H=Hres}, la valeur du paramètre de transmission \emph{S21} à la résonance \emph{f1} se rapproche le plus de celle à la résonance \emph{f2}. Lorsque la valeur du champs magnétique appliqué s'éloigne de \emph{Hres}, la valeur de \emph{S21} aux alentours de la résonance \emph{f1} se détériore.
\chapter{Etude du contrôle des pertes magnétiques dans le YIG}
Afin d'établir un contrôle sur les pertes magnétiques du YIG et de pouvoir évaluer la configurabilité de la résonance du filtre, un courant est injecté dans le platine. Ce courant injecté va être converti en onde de spin dans le YIG grâce à la combinaison des phénomènes de \emph{Spin Hall Effect} et de \emph{Spin Torque Transfer}\footnote{LETTERS Nature, \emph{Transmission of electrical signals by spin wave.}} et va modifier les propriétés résonantes du filtre.
\section{Temps de stabilisation des paramètres S du filtre configurable}
Cette section traite du temps mis par le prototype de filtre configurable pour se stabiliser lors de l'injection d'un courant dans le platine.
\noindent La figure~\underline{\color{blue}\ref{stabneg}} illustre le temps mis par le paramètre \emph{S21} pour se stabiliser lors de l'injection d'un courant dans le platine.
\begin{figure}[H]
	\centering
	\itshape
	\includegraphics[width=10cm,height=7cm]{stabneg.png}
	\caption{\label{stabneg} \underline{Temps de transition de S21 - 1}}
\end{figure}
\noindent Plus la valeur du courant injecté est élevé et plus le paramètre S21 va mettre du temps à atteindre sa valeur d'équilibre. En effet, par exemple le temps d'attente avant que S21 ne se stabilise pour un courant injecté dans le platine d'une valeur de \emph{20 mA} est d'environ \emph{3 s} tandis qu'il n'est que de quelques \emph{ms} pour \emph{I=75 mA}.
~\\\noindent La figure~\underline{\color{blue}\ref{stabpos}} illustre le temps mis par le paramètre \emph{S21} pour se stabilise lors de de l'injection d'un courant non-nul à un courant nul dans le platine.
\begin{figure}[H]
	\centering
	\itshape
	\includegraphics[width=10cm,height=7cm]{stabpos.png}
	\caption{\label{stabpos} \underline{Temps de transition de S21 - 2}}
\end{figure}
\noindent Le temps pour \emph{S21} d'atteindre sa position d'équilibre lors du passage d'un courant non-nul à un courant nul est plus élevé que dans le cas précédent (minimum environ \emph{3 s} pour \emph{I=20 mA}). Plus la valeur de courant injecté était grande et plus le paramètre S21 met du temps à se stabiliser.
\section{Balayage en courant et en fréquences}
\noindent Un champs magnétique \emph{H=Hres} est appliqué statiquement sur le filtre. A l'aide du logiciel \emph{Labview}, un balayage en courant avec un pas de \emph{2 mA} et en fréquences avec un pas de \emph{210 KHz} est effectué. Les résultats de mesures sur la valeur du paramètre \emph{S21} sont illustrés sur la figure~\underline{\color{blue}\ref{isweep}} suivante.
~\\\\Le paramètre de transmission \emph{S21} est d'autant plus faible sur les endroits de la courbe où la couleur est proche du bleu, c'est à dire aux fréquences auquelles le filtre va résonner.
~\\\\La fréquence de résonance \emph{f2} est la courbe rose la plus au dessus sur la figure~\underline{\color{blue}\ref{isweep}}. Sa valeur peut varier d'une centaine de \emph{MHz} en fonction du courant injecté.
~\\La fréquence de résonance \emph{f1} est la courbe rose la plus en dessous sur la figure~\underline{\color{blue}\ref{isweep}}. Plus la valeur du courant injecté dans le platine est importante et plus la valeur de \emph{f1} va s'éloigner fortement de la fréquence de résonance \emph{f0} du filtre à champs nul et sans courant injecté dans le platine.
\begin{figure}[H]
	\centering
	\itshape
	\includegraphics[width=15cm,height=10cm]{isweep.png}
	\caption{\label{isweep} \underline{Paramètre S21 - Balayage en courant et en fréquences}}
\end{figure}
\noindent\footnotesize\textbf{Figure~\underline{\color{blue}\ref{isweep}} | Balayage en courant et en fréquences. Courbes roses}, les courbes des minimums locaux du paramètre de transmission \emph{S21} aux alentours des fréquences \emph{f1} et \emph{f2}. \textbf{c et d}, les valeurs respectives du paramètre \emph{S21} aux alentours de \emph{f1} et  \emph{f2} pour une valeur de courant injecté dans le platine égale à \emph{18 mA}.
\normalsize
~\\\\L'interprétation des résultats de la mesure du balayage en courant et en fréquences est loin d'être évidente. En effet, la courbe de la résonance \emph{f1} est symétrique ce qui signifie que peu importe le sens du courant injecté dans le platine, la résonance \emph{f1} se décalera toujours dans le sens des fréquences inférieures. Alors qu'en injectant un courant de sens opposé dans le platine on aurait pu s'attendre à ce que le champs magnétique créé par le passage du courant dans le platine contribue à déplacer la résonance \emph{f1} soit positivement soit négativement. Or cette dernière se déplace toujours dans le même sens.
~\\\\Afin de pouvoir interprêter ses résultats, une étude supplémentaire autour de la contribution de l'injection d'un courant dans le platine et des phénomènes physiques complexes intervenant à l'interface entre le YIG et le platine est nécessaire.
\section{Correspondance des valeurs de S21 en champs et en courant}
~\\\noindent En observant les résultats de mesures en champs et en courant, il est possible d'établir une correspondance sur les valeurs de \emph{S21}.
~\\\\Les points \emph{C} et \emph{D} sur la figure~\underline{\color{blue}\ref{isweep}} correspondent aux valeurs des fréquences de résonance \emph{f1} et \emph{f2} pour une valeur de courant injecté \emph{i=18 mA}. Ces valeurs sont ensuite reportés sur la figure~\underline{\color{blue}\ref{matH}} pour obtenir les points \emph{C} et \emph{D}. Ceux-ci correspondent à une valeur de champs magnétique \emph{H=970.75 Oe}.
~\\\\La figure~\underline{\color{blue}\ref{corresp}} suivante illustre l'allure de \emph{S21} pour \emph{H=970.75 Oe} et pour \emph{i=18 mA}.
\begin{figure}[H]
	\centering
	\itshape
	\includegraphics[width=15cm,height=10cm]{corresp.png}
	\caption{\label{corresp} \underline{Correspondance de S21 en champs et en courant}}
\end{figure}
\noindent La similitude entre les deux courbes est clairement établie. Néanmoins, il y a un léger décalage présent autour de la fréquence de résonance \emph{f1}. Ceci peut-être dû à la variation du gradient de température dans le YIG induite par l'injection d'un courant dans le platine pouvant modifier les propriétés du YIG comme son aimantation à saturation \emph{Ms}.
Pour \emph{I=18 mA} et \emph{H=Hres}, la valeur du paramètre \emph{S21} atteint des minimums locaux plus important autour des fréquences de résonance \emph{f1} et \emph{f2} (entre 1 et 2 dB de différence) que pour la mesure de \emph{S21} à \emph{H=970.75 Oe}.
\section{Détermination du coefficient de couplage}
~\\\noindent Les pertes magnétiques sont reliées au gap de fréquences entre les fréquences de résonance \emph{f1} et \emph{f2} par la formule suivante\footnote{Article, \emph{Control of Magnon-Photon Coupling Strength in a Planar Resonator/Yttrium-Iron-Garnet Thin-Film Configuration}}: 
$$\frac{\underset{1}{f}-\underset{2}{f}}{2}\ \text{=}\ \frac{\underset{eff}{g}}{2*\pi}$$
avec $\frac{\underset{eff}{g}}{2*\pi}$ lié au coefficient de couplage \emph{k}.
~\\\\La figure~\underline{\color{blue}\ref{coeffcoupl}} illustre les valeurs prises par $\frac{\underset{eff}{g}}{2*\pi}$ en fonction du courant injecté dans le platine.
\begin{figure}[H]
	\centering
	\itshape
	\includegraphics[width=15cm,height=10cm]{coeffcoupl.png}
	\caption{\label{coeffcoupl} \underline{Coefficient du couplage électrique et magnétique}}
\end{figure}
\noindent A partir d'une valeur de courant injecté dans le platine supérieur à environ 25 mA, le coefficient de couplage se dégrade rapidement jusqu'à atteindre une valeur nulle à 50 mA. Les barres d'incertitudes illustre la reproductibilité de la mesure. L'incertitude sur le coefficient de couplage sera d'autant plus élevée aux alentours de 50 mA ainsi qu'aux alentours du \emph{glitch} de mesure entre 25 et 30 mA présent sur les mesures de balayage en courant et en fréquences (voir figure~\underline{\color{blue}\ref{isweep}}).

\chapter{Etude de la détection de la tension DC en bout de chaîne}
Ce chapitre traite de l'étude de la détection de la tension DC en bout de chaîne à la condition de résonance du YIG pour une valeur de champs magnétique \emph{H=Hres} ainsi que de l'influence du courant injecté dans le platine sur la tension DC récupérée en bout de chaîne.
~\\\\ Les mesures en tension sont effecutées aux bornes du platine en augmentant progressivement la fréquence de 4.4 GHz à 5.1 GHz avec une précision de 5000 points de mesures. Les mesures sont ensuite récupérées dans un fichier à l'aide du logiciel \emph{Labview}.
\section{Mesures à champs nul}
\noindent La figure~\underline{\color{blue}\ref{dcnull}} illustre la tension DC en bout de chaîne à champs magnétique nul.
\begin{figure}[H]
	\centering
	\itshape
	\includegraphics[width=8cm,height=6cm]{dcnull.png}
	\caption{\label{dcnull} \underline{Tension DC à H=0 Oe}}
\end{figure}
\section{Mesures à champs magnétique positif}
\noindent La figure~\underline{\color{blue}\ref{dchpos}} illustre la tension DC en bout de chaîne à champs magnétique \emph{H=+Hres}.
\begin{figure}[H]
	\centering
	\itshape
	\includegraphics[width=10cm,height=7cm]{dchpos.png}
	\caption{\label{dchpos} \underline{Tension DC à H=+Hres}}
\end{figure}
\section{Mesures à champs magnétique négatif}
\noindent La figure~\underline{\color{blue}\ref{dchneg}} illustre la tension DC en bout de chaîne à champs magnétique \emph{H=-Hres}.
\begin{figure}[H]
	\centering
	\itshape
	\includegraphics[width=10cm,height=7cm]{dchneg.png}
	\caption{\label{dchneg} \underline{Tension DC à H=-Hres}}
\end{figure}
\section{Interprétation des résultats de mesures de la tension DC}
\noindent A champs nul, la tension DC détectée n'est autre que le bruit généré par l'environnement de mesures à l'air libre et l'application du signal RF dans le filtre.
~\\Pour une valeur de champs positive \emph{H=+Hres}, la tension détectée est négative tandis qu'à \emph{H=-Hres}, la tension détectée est positive. Dans ces deux cas, la tension détectée est de l'ordre du micro-volt.
~\\Lors de l'application d'un champs magnétique \emph{H}=$\pm$\emph{Hres}, le pic de tension le plus élevé est situé à la fréquence de résonance magnétique \emph{fr} du YIG comme illustré sur la figure~\underline{\color{blue}\ref{dcfr}}:
\begin{figure}[H]
	\centering
	\itshape
	\includegraphics[width=15cm,height=6cm]{dcfr3.png}
	\caption{\label{dcfr} \underline{Tension DC et S21 à H=Hres}}
\end{figure}
\noindent Les différents pics de tensions apparant autres que celui centré à la fréquence de résonance \emph{fr} du YIG sont dûs à la résonance des autres modes que le mode fondamentale dans le YIG. 
~\\\\L'apparition d'une tension DC en bout de chaîne est dûe à l'apparition d'un courant de spin dans le YIG à sa condition de résonance magnétique \emph{H=Hres}. Ce courant de spin est ensuite converti en courant électrique dans le platine par le biais des phénomènes physiques de \emph{Spin-Pumping} et de \emph{Inverse Spin Hall Effect}\footnote{Letters, \emph{Transmission of electrical signals by spin-wave interconversion in a magnetic insulator}} ayant lieu à l'interface entre le YIG et le platine.
\chapter{Influence du courant injecté dans le platine}
Des mesures ont été effectuées pour déterminer la réponse en tension lors de l'injection d'un courant dans le platine et pour essayer de comprendre le phénomène physique responsable du décalage de la fréquence de résonance \emph{f1} s'effectuant toujours dans le même sens peu importe le sens du courant injecté. 
\section{Courbe d'hystéresis}
~\\\noindent La figure~\underline{\color{blue}\ref{hyst}} illustre la tension détectée aux bornes du platine lors de l'injection d'un courant dans ce dernier. Les mesures ont été effectuées de \emph{-50 mA} à {+50 mA} et inversement pour obtenir le cycle d'hystéresis total avec une mesure tous les \emph{5 mA}.
\begin{figure}[H]
	\centering
	\itshape
	\includegraphics[width=10cm,height=7cm]{hyst.png}
	\caption{\label{hyst} \underline{Courbe d'hystéresis}}
\end{figure}
\noindent Entre \emph{-20 mA} et \emph{+20 mA} la réponse en tension est linéaire. Mais au delà de cette intervalle, la réponse en tension a un caractère non linéaire et ne peut donc pas être modélisé par la formule \emph{U=R.I}.
\section{Impédance - Rapport courant/tension}
~\\\noindent La figure~\underline{\color{blue}\ref{imp}} illustre le rapport de la tension U/I dans le sens montant (mesures effectuées avec un pas positif) et dans le sens descendant (mesures effectuées avec un pas négatif).
\begin{figure}[H]
	\centering
	\itshape
	\includegraphics[width=10cm,height=7cm]{imp.png}
	\caption{\label{imp} \underline{Rapport courant/tension}}
\end{figure}
La non-linéarité se retrouve également sur la figure ci-dessus.
\section{Contribution en champs magnétique du courant injecté dans le platine}
~\\\noindent Un article scientifique\footnote{Physical Review, \emph{Nanosecond time-scale switching of permalloy thin film elements studied by wide-field time-resolved Kerr microscopy}} détail le modèle analytique du champs magnétique généré par l'injection d'un courant dans du platine.
\subsection{La composante Hx parallèle à l'élément platine}
\noindent La figure~\underline{\color{blue}\ref{hx}} illustre la composante analytique \emph{Hx} généré par l'injection d'un courant dans le platine en fonction de la distance par rapport à ce dernier.
\begin{figure}[H]
	\centering
	\itshape
	\includegraphics[width=8cm,height=7cm]{Hx.png}
	\caption{\label{hx} \underline{Composante Hx du champs magnétique généré par l'injection d'un courant}}
\end{figure}
\subsection{La composante Hy perpendiculaire à l'élément platine}
\noindent La figure~\underline{\color{blue}\ref{hy}} illustre la composante analytique \emph{Hy} perpendiculaire à l'élément platine généré par l'injection d'un courant dans le platine.
\begin{figure}[H]
	\centering
	\itshape
	\includegraphics[width=15cm,height=7cm]{Hy.png}
	\caption{\label{hy} \underline{Composante Hy du champs magnétique généré par l'injection d'un courant}}
\end{figure}
\subsection{Interprétation de la contribution du courant injecté sur les valeurs en champs magnétique }
\noindent Afin d'observer la contribution du courant injecté dans le platine sur le champs magnétique généré, la figure~\underline{\color{blue}\ref{contribh}} illustre les maximums des valeurs des composantes de champs \emph{Hx} et \emph{Hy}.
\begin{figure}[H]
	\centering
	\itshape
	\includegraphics[width=10cm,height=7cm]{contribh.png}
	\caption{\label{contribh} \underline{Contribution en champs magnétique du courant injecté dans le platine}}
\end{figure}
\noindent La contribution en champs magnétique lors de l'injection d'un courant dans un élément de platine d'épaisseur \emph{6nm}, est de \emph{0.32 Oe} pour la composante \emph{Hx} et de \emph{1.2 Oe} au maximum pour la composante \emph{Hy} à l'interface du platine.
~\\\\Ceci permet de montrer que la contribution en champs magnétique de l'injection d'un courant dans le platine n'est pas le phénomène physique responsable de la variation du champs magnétique car cette contribution est minimale. Ceci permet d'éliminer l'hypothèse que selon la valeur et le sens du courant injecté dans le platine, la fréquence de résonance \emph{f1} se décale vers la droite ou vers la gauche car dans les faits, cette dernière se décale toujours vers les fréquences plus faibles peu importe le sens du courant.
De plus, comme cela a été vu dans la partie~\underline{\color{blue}3.2}, pour une valeur injecté de \emph{18 mA} à \emph{H=Hres=990 Oe}, la correspondance se situe pour une valeur de champs de \emph{H=970 Oe} à \emph{I=O mA}. Le courant injecté dans le platine entraîne donc une différence de \emph{20 Oe} dans ce cas de figure ce qui est bien supérieur à la contribution maximale du champs magnétique généré par l'injection d'un courant dans le platine.
~\\\\Afin de déterminer les causes d'une telle variation sur le champs magnétique total lors de l'injection d'un courant dans le platine, plusieurs autres hypothèses restent à étudier pour comprend parfaitement le phénomène telles que la variation du gradient de température dans le YIG lors de l'injection d'un courant dans le platine et la variation des paramètres caractéristiques (aimentation à saturation, etc.) du YIG induite par ce courant. Ces hypothèses n'ont pas eu le temps d'être vérifiées au cours de ce projet de 6 mois mais peuvent constituer une piste de recherche sèrieuse pour la compréhension
du phénomène de \emph{splitting} des résonances dans son ensemble.
\part{Conception d'un boîtier sous ADS}
\part{Conclusion}

\setcounter{part}{-5}
\part{Annexes}

\setcounter{part}{-6}
\part{Bibliographie}
\label{Sources}
\appendix
\begin{description}
 \item[$\bullet$] ArXiv, \emph{Control of magnon-photon coupling strength in a planar resonator/YIG thin film configuration}
 \item[$\bullet$] LETTERS, Nature, \emph{Transmission of electrical signals by spin wave}
 \item[$\bullet$] Arxiv, \emph{Study of the cavity-magnon-polariton transmission line shape}.
\end{description}

\setcounter{part}{-7}
\part{Glossaire}
\begin{description}
 \item[$\bullet$] YIG: \emph{Yittrium Iron Garnet}
 \item[$\bullet$] Pt: \emph{Platine}
\end{description}


\end{document}
